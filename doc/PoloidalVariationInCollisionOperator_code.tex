
\documentclass[12pt]{article}
\usepackage{url} 
%\usepackage[dvips]{graphicx}
\usepackage[pdftex]{graphicx}
\usepackage[latin1]{inputenc}
\usepackage{amsmath}
\usepackage{amssymb}
\usepackage{fancyhdr}
\usepackage{bm}
\usepackage{float}
\usepackage{listings}
\usepackage{color}

\definecolor{dkgreen}{rgb}{0,0.6,0}
\definecolor{gray}{rgb}{0.5,0.5,0.5}
\definecolor{mauve}{rgb}{0.58,0,0.82}

\lstset{frame=tb,
  language=Fortran,
  aboveskip=3mm,
  belowskip=3mm,
  showstringspaces=false,
  columns=flexible,
  basicstyle={\small\ttfamily},
  numbers=none,
  numberstyle=\tiny\color{gray},
  keywordstyle=\color{blue},
  commentstyle=\color{dkgreen},
  stringstyle=\color{mauve},
  breaklines=true,
  breakatwhitespace=true,
  tabsize=3,
  frame=none
}\usepackage[dvipsnames]{xcolor}
\usepackage{wrapfig}
\usepackage{multicol}
\usepackage[colorlinks=true,
		linkcolor=red,
		citecolor=blue,
		urlcolor=blue]{hyperref}

\usepackage{titlesec}
\usepackage[titletoc]{appendix}
%\usepackage{appendix}
\setcounter{secnumdepth}{4}

\titleformat{\paragraph}
{\normalfont\normalsize\bfseries}{\theparagraph}{1em}{}
\titlespacing*{\paragraph}
{0pt}{3.25ex plus 1ex minus .2ex}{1.5ex plus .2ex}

\setlength {\parindent} { 10mm} 
\setlength{\textheight}{230mm} 
\setlength{\textwidth}{160mm} 
\setlength{\oddsidemargin}{0mm}
\setlength{\topmargin}{-10mm} 
% newcommands
\newcommand{\p}{\partial}
\newcommand{\g}[1]{\mbox{\boldmath $#1$}}
\newcommand{\vi}{\g V_{\! \! i}}
\newcommand{\ps}{Pfirsch-Schl\"{u}ter} 
\newcommand{\lp}{\left(}
\newcommand{\rp}{\right)}
\newcommand{\ca}[1]{\mbox{\cal $#1$}}
\newcommand{\be}{\begin{displaymath}}
\newcommand{\ee}{\end{displaymath}}
\newcommand{\bn}{\begin{equation}}
\newcommand{\en}{\end{equation}}
\newcommand{\mygtrsim}{\mathrel{\mbox{\raisebox{-1mm}{$\stackrel{>}{\sim}$}}}}
\newcommand{\mylsim}{\mathrel{\mbox{\raisebox{-1mm}{$\stackrel{<}{\sim}$}}}}
\newcommand{\vek}{\bf}
\newcommand{\ten}{\sf}
\newcommand{\bfm}[1]{\mbox{\boldmath$#1$}}
\newcommand{\lang}{\left\langle}
\newcommand{\rang}{\right\rangle}
\newcommand{\vo}[1]{\left|\begin {array}{l} \mbox{} \\ \mbox{} \\$#1$ \end
{array}\right .}  
\newcommand{\von}[2]{\left |\begin {array}{l}
\mbox{}\\$#1$\\$#2$ \end {array}\right .}
\newcommand{\simgt}{\:{\raisebox{-1.5mm}{$\stackrel
{\textstyle{>}}{\sim}$}}\:}
\newcommand{\simlt}{\:{\raisebox{-1.5mm}{$\stackrel
{\textstyle{<}}{\sim}$}}\:}
%\renewcommand {\baselinestretch} {1.67}
%\pagestyle{empty}
\newcommand{\todo}[1]{\textbf{\textcolor{red}{TODO: #1}}}
\newcommand{\remark}[1]{\textbf{\textcolor{red}{REMARK: #1}}}

\title{Implementation of poloidal density variation in collision operator}

\pagestyle{fancy}
\fancyhead{}
\chead{Aylwin Iantchenko %850227-2019
\\ Implementation of poloidal density variation in collision operator}
\cfoot{\thepage}
\renewcommand{\headrulewidth}{1pt}
\renewcommand{\footrulewidth}{1pt}
\setlength{\headheight}{28pt}
\setlength{\footskip}{25pt}

\newcommand{\red}[1]{\textcolor{red}{#1}}
\newcommand{\mE}{\mathcal{E}}
\newcommand{\energy}{\mathcal{E}}
\newcommand{\mK}{\mathcal{K}}
\newcommand{\mN}{\mathcal{N}}
\newcommand{\mD}{\mathcal{D}}
\newcommand{\ord}{\mathcal{O}}
\newcommand{\Tpe}{T_\perp}
\newcommand{\Tpa}{T_\|}
\newcommand{\vpe}{v_\perp}
\newcommand{\vpa}{v_\|}
\newcommand{\kpe}{k_\perp}
\newcommand{\kpa}{k_\|}
\newcommand{\Bv}{\mathbf{B}}
\newcommand{\Ev}{\mathbf{E}}
\newcommand{\bv}{\mathbf{b}}
\newcommand{\vv}{\mathbf{v}}
\newcommand{\cd}{\cdot}
\newcommand{\na}{\nabla}
\newcommand{\btheta}{\bar{\theta}}
\newcommand{\phit}{\tilde{\phi}}
\newcommand{\oert}{\tilde{\omega}_{Er}}

% Simplify equations
\newcommand{\eq}[1]{\begin{align*}\begin{gathered}#1\end{gathered}\end{align*}}
\newcommand{\eqre}[1]{\begin{align}\begin{gathered}#1\end{gathered}\end{align}}

\begin{document}
\titlepage

\maketitle

The aim with this document is to explain the modifications in \texttt{SFINCS} that have been done, in order to include poloidal density variation in the Fokker-Planck collision operator. 

As is explained in ref.~\cite{ref:PolVarColl_stefan} poloidal density variation can be included by modifying the lowest order distribution function 
\eq{
f_{Ms}(\psi) \rightarrow f_{Ms}(\psi)e^{-Z_se\Phi_1(\theta,\zeta)/T_s},
}
where $f_M(\psi)$ is the flux-function Maxwellian, $\Phi_1$ are first order variation of the electrostatic potential, and $Z_s,T_s$ are the charge and temperature of species $s$ respectively. To implement this change one may simply modify the species density accordingly
\eq{
n_s \rightarrow n_se^{-Ze\Phi_1(\theta,\zeta)/T},
}
in the original equations for the various terms in the collision operator, described in ref.~\cite{ref:Coll1}.

The linearised collision operator takes the form
\eqre{
C_{ab}^{L:f0} =  C_{ab}\left\{f_{aM},f_{bM}\right\}e^{-\left(Z_a/T_a + Z_b/T_b\right)e\Phi_1(\theta,\zeta)} + C_{ab}\left\{f_{a1},f_{bM}\right\}e^{-Z_b/T_be\Phi_1(\theta,\zeta)} + \\ + C_{ab}\left\{f_{aM},f_{b1}\right\}e^{-Z_a/T_ae\Phi_1(\theta,\zeta)},
}
which describes collisions between species $a$ and $b$. In normalised \texttt{SFINCS} units this becomes
\eqre{
\label{eq:CollMod}
\hat C_{ab}^{L:f0} = \hat C_{ab}\left\{f_{aM},f_{bM}\right\}e^{-\left(\hat Z_a/\hat T_a + \hat Z_b/\hat T_b\right) \alpha\hat\Phi_1(\theta,\zeta)} + \hat C_{ab}\left\{f_{a1},f_{bM}\right\}e^{-\hat Z_b/\hat T_b \alpha\hat\Phi_1(\theta,\zeta)} + \\ + \hat C_{ab}\left\{f_{aM},f_{b1}\right\}e^{-\hat Z_a/\hat T_a\alpha\hat\Phi_1(\theta,\zeta)}.
}

\noindent
The first term on the RHS of Eq.~\eqref{eq:CollMod} is the \textit{temperature equilibration term}. Since this term does not include $f_{1s}$ it has to be treated differently compared to the other two cases. 

The contribution to the Jacobian are $\partial \hat C_{ab}^{L:f0}/\partial \hat f_{1s}$ and $\partial \hat C_{ab}^{L:f0}/\partial \hat \Phi_1$. The first is the same as in Eq.~\eqref{eq:CollMod} except that the collision operator does not act on the distribution function. In the second case we get 

\eqre{
\label{eq:Jac1}
\hat C_{ab}^{L:f0} = -\left(\hat Z_a/\hat T_a + \hat Z_b/\hat T_b\right) \alpha \hat C_{ab}\left\{f_{aM},f_{bM}\right\}e^{-\left(\hat Z_a/\hat T_a + \hat Z_b/\hat T_b\right) \alpha\hat\Phi_1(\theta,\zeta)}  \\ -\hat Z_b/\hat T_b \alpha\ \hat C_{ab}\left\{f_{a1},f_{bM}\right\}e^{-\hat Z_b/\hat T_b \alpha\hat\Phi_1(\theta,\zeta)}  - \hat Z_a/\hat T_a\alpha\hat\Phi_1(\theta,\zeta) \hat C_{ab}\left\{f_{aM},f_{b1}\right\}e^{-\hat Z_a/\hat T_a\alpha\hat\Phi_1(\theta,\zeta)}.
}

\noindent
Apart from the different prefactor, the term in Eq.~\eqref{eq:Jac1} differs from term in $\partial \hat C_{ab}^{L:f0}/\partial \hat f_{1s}$ by acting on the first order distribution function $f_{1s}$. For example, in the first case, terms such as $\partial f_{1s}/\partial x_s$ become 
\eq{
\frac{1}{\partial f_{1s}}\frac{\partial f_{1s}}{\partial x_s} = \frac{\partial}{\partial x_s},
}
whereas in the second case, when take the derivative with respect to $\Phi_1$, we have to actually calculate the derivative $\partial f_{1s}/\partial x_s$.

%We will now go through the various changes in the code that have been done to implement Eq.~\eqref{eq:CollMod}. 

\section*{Implementation in the Code}
To implement Eq.~\eqref{eq:CollMod} we need to modify the evaluation of the residual and the jacobian. The residual is evaluated in \texttt{evaluateResidual.F90} and the jacobian in \texttt{evaluateJacobian.F90}. Both these two routines call the functions in \texttt{PopulateMatrix.F90}, where the actual assembly of the residual and jacobian matrices are done. Therefore, we only need to modify \texttt{PopulateMatrix.F90}.


\subsection*{Residual}
\label{sec:Res}
In the residual we have to include the extra pre-factor appearing in Eq.~\eqref{eq:CollMod}. This is done by modifying the last block in the calculation of the collision operator, just before the terms are saved into the main matrix. We introduce the pre-factor \texttt{preFactor} which is normally equal to one, but $e^{-\hat Z_a/\hat T_a \alpha\hat\Phi_1(\theta,\zeta)}$ when poloidal density variation is included (by setting the flag \texttt{poloidalVariationInCollisionOperator = .true.}). 

\begin{lstlisting}
  preFactor = 1.0 ! Initiate the preFactor used to multiply CHat before saving into the Main matrix
  if (poloidalVariationInCollisionOperator) then 
     preFactor = exp(-Zs(iSpeciesA)*alpha*Phi1Hat(itheta,izeta)/Thats(iSpeciesA)) 
\end{lstlisting}
This factor then multiplies the collision operator, just before saving into the main matrix
\begin{lstlisting}
  call MatSetValueSparse(matrix, rowIndex, colIndex, &
       -nu_n*preFactor*CHat(ix_row,ix_col), ADD_VALUES, ierr) ! Multiply CHat with preFactor before saving
\end{lstlisting}
\subsubsection*{Temperature equilibration term}
In the original version of the code, the temperature equilibration term is calculated in \\ \texttt{evaluateResidual.F90} by using \texttt{whichMatrix = 2} when calling \texttt{PopulateMatrix.F90}. The output is then multiplied with the first order distribution function $f_{0s}$. Since the original version of \texttt{SFINCS} already includes $\Phi_1$ in the definition of $f_{0s}$ (in the subroutine \texttt{init\_f0()} inside \texttt{PopulateMatrix.F90}), the same routines as were presented above can be used. In the end, when calling \texttt{PopulateMatrix} we include the pre-factor $e^{-\hat Z_a/\hat T_a \alpha\hat\Phi_1(\theta,\zeta)}$. Then by multiplying with $f_{0b}$ in \texttt{evaluateResidual.F90} we get the desired total pre-factor $e^{-\left(\hat Z_a/\hat T_a + \hat Z_b/\hat T_b\right) \alpha\hat\Phi_1(\theta,\zeta)}$.

\subsection*{Jacobian}
The $\partial C/\partial f_{1s}$ contribution in the jacobian is at this point already taking the density variations into account if we implement the modifications explained in Section~\ref{sec:Res}. However, we need to add the calculation of $\partial C/\partial \Phi_1$.

\subsubsection*{The $\partial C/\partial \Phi_1$ term}
\label{sec:Jac1}
To implement the derivative with respect to $\Phi_1$ we add another block right after the calculation of the residual terms. This block should be evaluated whenever the jacobian (\texttt{whichmatrix =1}) or the preconditioner (\texttt{whichmatrix =0}) is calculated, with \texttt{poloidalVariationInCollisionOperator = .true.}

\begin{lstlisting}
 if (poloidalVariationInCollisionOperator .and. (whichMatrix == 1 .or. whichMatrix == 0)) then
\end{lstlisting}

We begin with treating the other two terms appearing in Eq.~\eqref{eq:Jac1} and will then look at the temperature equilibration term. For the second two terms in Eq.~\eqref{eq:Jac1} we first have to implement the correct pre-factor.

\begin{lstlisting}
 preFactor = -Zs(iSpeciesA)*alpha/Thats(iSpeciesA)*exp(-Zs(iSpeciesA)&
 *alpha*Phi1Hat(itheta,izeta)/Thats(iSpeciesA))\end{lstlisting}

\noindent
The second change is including the first order distribution function (and calculating derivatives thereof). For this purpose we first have to initiate this distribution function, by reading the appropriate terms in the current state vector 

\begin{lstlisting}
 index = getIndex(iSpeciesB,ix,L+1,itheta,izeta,BLOCK_F) ! f1b from statevector
 f1b(ix) = stateArray(index + 1)
\end{lstlisting}
\noindent
What is left is to multiply this vector with the collision operator matrix \texttt{CHat} containing terms such as $\partial/\partial x_s$.

 \begin{lstlisting}
 CHatTimesf = matmul(CHat,f1b)
 \end{lstlisting}
\noindent
The output is then saved into the main matrix in the $\partial \Hat C/\partial \hat \Phi_1$ block,

 \begin{lstlisting}
 call MatSetValue(matrix, rowIndex, colIndex, & 
 -nu_n*preFactor*CHatTimesf(ix_col), ADD_VALUES, ierr) 
\end{lstlisting}
\noindent
using the approptiate row and column indices.

 \begin{lstlisting}
rowIndex=getIndex(iSpeciesA,ix_row,L+1,itheta,izeta,BLOCK_F)
do ix_col = max(ixMinCol,min_x_for_L(L)),Nx
   ! Get column index for the d/dPhi1 terms
   colIndex=getIndex(1,1,1,itheta,izeta,BLOCK_QN)
\end{lstlisting}


\subsubsection*{Temperature equilibration}
\label{sec:Jac2}
If the temperature equilibration terms should be included, we have to add additional terms into the main matrix, when calculating  $\partial \Hat C/\partial \hat \Phi_1$. To begin with we need to include the different pre-factor (since in this case $f_0$ from \texttt{init\_f0()} is not used).

\begin{lstlisting}
if (includeTemperatureEquilibrationTerm) then
   ! Get the correct preFactor
   preFactor = (-Zs(iSpeciesA)*alpha/Thats(iSpeciesA) -Zs(iSpeciesB)*alpha/  & 
    Thats(iSpeciesB)) 
   *exp(-Zs(iSpeciesA)*alpha*Phi1Hat(itheta,izeta)/Thats(iSpeciesA)-Zs(iSpeciesB) & 
   *alpha*Phi1Hat(itheta,izeta)/Thats(iSpeciesB))
\end{lstlisting}
\noindent
The second change originates from the fact that the collision operator in this temperature equilibration case, should multiply the Maxwellian for species $b$ and not $f_{1b}$ (as was previously the case in Section~\ref{sec:Jac2}).  First, we have to initiate this distribution function
 \begin{lstlisting}
 if (includeTemperatureEquilibrationTerm) then
    fM(ix) = expxb2(ix)*nhats(iSpeciesB)*mhats(iSpeciesB)*sqrt(mhats(iSpeciesB)) &
     /(pi*sqrtpi*Thats(iSpeciesB)*sqrt(Thats(iSpeciesB)))
     end if   
\end{lstlisting}
\noindent
and then use it to multiply to the total collision operator.
 \begin{lstlisting}
  CHatTimesf = matmul(CHat,fM)
 \end{lstlisting}
\noindent
Then we add this term into the main matrix, using the the same procedure as was explained in Section~\ref{sec:Jac1}

\newpage

\begin{thebibliography}{99}

\bibitem{ref:PolVarColl_stefan} S.~Buller, {\em Poloidal variation in collision operator} (2017).

\bibitem{ref:Coll1} M.~Landreman, {\em Technical Documentation for version 3 of SFINCS} (2014).

%\bibitem{simakov} A.~N.~Simakov, P.~Helander,
%  {\em Phys. Plasmas} {\bf 16}, 042503 (2009).
  
%\bibitem{nonAxis} P.~Helander, Theory of plasma confinement in non-axisymmetric magnetic fields (2013).

%\bibitem{MH} L.~Råde and B.~Westergren, Mathematics Handbook for Science and Engineering, $5^{\mathrm{th}}$ edition, 2004. %\vspace{-5mm}

%\bibitem{Abra} M.~Abramowitz and I.~A.~Stegun, Handbook of Mathematical Functions, $10^{\mathrm{th}}$ printing, 1972.

\end{thebibliography}

\end{document}
