\chapter{Output Parameters}
\label{ch:output}

\newcommand{\outputparam}[4]{{\setlength{\parindent}{0cm} {\ttfamily \bfseries \hypertarget{#1}{#1}}\\{\it Type}: #2\\{\it Available when}: #3\\{\it Meaning}: #4}}
\newcommand{\outputparamWithUnderscore}[5]{{\setlength{\parindent}{0cm} {\ttfamily \bfseries \hypertarget{#2}{#1}}\\{\it Type}: #3\\{\it Available when}: #4\\{\it Meaning}: #5}}
%\newcommand{\underscoreparlink}[2]{{\ttfamily \hyperlink{#2}{#1}}} %%Moved this to SFINCSUserManual.tex

In this chapter we describe output parameters which are available in the \sfincs~output file (specified by \parlink{outputFileName}) after a %successful 
\sfincs~run. 
The output is in the \HDF~format, and the output can be looked at e.g. by using the {\ttfamily h5dump} command on Linux. 
Note that all parameters in the output file are case-sensitive. 
Moreover, many of the input parameters in chapter~\ref{ch:input} also exist in the output file and we refrain from repeating them here. 


\section{Convergence and test parameters}
\label{sec:ConvergenceAndTestParameters}

\myhrule

\outputparam{finished}
{integer}
{\sfincs~has finished.}
{If this variable exists, then \sfincs~reached the end of all requested computations and exited gracefully. 
Note that this does not guarantee that a nonlinear calculation (\parlink{includePhi1} == \true) has converged.}

\myhrule

\outputparam{integerToRepresentTrue}
{integer}
{Always}
{Since \HDF~does not have a Boolean datatype, this integer value is used to represent \true.}

\myhrule

\outputparam{integerToRepresentFalse}
{integer}
{Always}
{Since \HDF~does not have a Boolean datatype, this integer value is used to represent \false.}

\myhrule

\outputparam{didNonlinearCalculationConverge}
{integer}
{\parlink{RHSMode} == 1, \parlink{solveSystem} == \true, \parlink{includePhi1} == \true, \parlink{readExternalPhi1} == \false and \parlink{finished} available.}
{If this variable is \true~the nonlinear iteration has converged. Interpret this Boolean data by comparing to \parlink{integerToRepresentTrue}/\parlink{integerToRepresentFalse}.}

\myhrule

\outputparam{lambda}
{real}
{\parlink{RHSMode} == 1, \parlink{solveSystem} == \true, \parlink{includePhi1} == \true and \parlink{readExternalPhi1} == \false}
{Lagrange multiplier associated with the constraint that $\left\langle \Phi_1\right\rangle = 0$, which is included in the quasineutrality equation. %Should be within machine precision of 0.
This parameter should come out as negligible in quasineutrality, e.g. its size should be significantly smaller than the maximum charge density of the plasma species $Z_s n_s$.
%\todo{Is this correct?}
}

\myhrule

\outputparam{NIterations}
{integer}
{\parlink{RHSMode} == 1 and \parlink{solveSystem} == \true}
{Number of iterations in the nonlinear calculation. (In a linear calculation this will be 1.)}

\myhrule



\section{Radial fluxes and flows}
\label{sec:RadialFluxesAndFlows}

A detailed description of the most important output fluxes and flows are given in the Technical Documentation for \sfincs. 
Here we give a concise version of some of the quantities. 
Transforming the dimension of the output quantities into SI units is typically done with the normalizations 
$\bar{B}$, $\bar{R}$, $\bar{n}$, $\bar{m}$, $\bar{T}$, $\bar{v}$, $\bar{\Phi}$ (see section~\ref{sec:normalizations}). 

\subsubsection{Radial fluxes}

The radial fluxes have the form 
\[
\left\langle \bm{X} \cdot \nabla Y \right\rangle
\]
where $Y$ is a flux-surface label and $\left< \ldots \right>$ denotes a flux surface average. In \sfincs~there are 4 possible labels: $\hat{\psi}$, $\psi_N$, $\hat{r}$ and $r_N$ (see section~\ref{sec:radialCoordinates}), indicated by the suffix {\ttfamily \_psiHat}, {\ttfamily \_psiN}, {\ttfamily \_rHat} or {\ttfamily \_rN} in the output variable name. 
Below we will only present the description of the {\ttfamily XXX\_psiHat} variables, but the description of the other variables are readily obtained by noting that 
all flux-surface labels are dimensionless which implies that the normalization factor will be the same. E.g. when $\displaystyle \left\langle \bm{X} \cdot \nabla \hat{\psi} \right\rangle = \text{\ttfamily NormalizationFactor} \cdot \text{\ttfamily XXX\_psiHat}$ then also $\displaystyle \left\langle \bm{X} \cdot \nabla \hat{r} \right\rangle = \text{\ttfamily NormalizationFactor} \cdot \text{\ttfamily XXX\_rHat}$ applies with the same {\ttfamily NormalizationFactor}. 

For the radial flux of each quantity (particle, momentum and energy), and for each radial coordinate, there is an output quantity associated with the zeroth-order $f_{s0}$ or with the full distribution function $f_s = f_{s0} + f_{s1}$, and there is also a quantity associated with the radial magnetic drift, with the radial $\bm{E} \times \bm{B}$-drift or with the total drift $\bm{v}_{ds} = \bm{v}_{ms} + \bm{v}_{E}$. Here we only present all possible combinations for the particle flux, but the pattern is the same for the momentum and energy flux. 
%
\subsubsection{Classical fluxes}
%
The classical flux is often neglected as smaller than the neoclassical flux. However, in a stellarator that has been optimized to have a low ratio of parallel to perpendicular current (like W7-X), the classical flux can be of the same size as the neoclassical (particularly if the collisionality is large) \cite{Buller2018}. 
\sfincs~calculates the radial classical particle and heat fluxes, using the same normalizations as for the corresponding neoclassical fluxes. 
Note that the classical fluxes (in contrast to the neoclassical) depend on the gyro-phase dependent part of the distribution function. 
However, the first-order gyro-phase dependent part of the distribution function does not depend on the solution to the drift-kinetic equation in \sfincs, and the classical fluxes can thus be calculated to required order even without solving it. 
The classical fluxes do depend on $\Phi_1$ so if the quasi-neutrality equation is solved the fluxes will depend on the solution to the system of equations. 

\myhrule

\outputparamWithUnderscore{particleFlux\_vd\_psiHat}
{particleFlux_vd_psiHat}
{%$\text{\ttfamily Nspecies} \times \text{\ttfamily NIterations}$ 
\parlink{Nspecies}~$\times$~\parlink{NIterations} matrix of reals}
{\parlink{RHSMode} == 1, \parlink{solveSystem} == \true and \parlink{includePhi1} == \true}
{$\displaystyle \text{\ttfamily particleFlux\_vd\_psiHat} = \frac{\bar{R}}{\bar{n} \bar{v}} \left\langle \int d^3v f_s \, \bm{v}_{ds} \cdot \nabla \hat{\psi}\right\rangle$ for all plasma species. The final result is in the last index corresponding to the last nonlinear iteration.}

\myhrule

\outputparamWithUnderscore{particleFlux\_vm\_psiHat}
{particleFlux_vm_psiHat}
{\parlink{Nspecies}~$\times$~\parlink{NIterations} matrix of reals}
{\parlink{RHSMode} == 1 and \parlink{solveSystem} == \true}
{$\displaystyle \text{\ttfamily particleFlux\_vm\_psiHat} = \frac{\bar{R}}{\bar{n} \bar{v}} \left\langle \int d^3v f_s \, \bm{v}_{ms} \cdot \nabla \hat{\psi}\right\rangle$ for all plasma species. The final result is in the last index corresponding to the last nonlinear iteration.}

\myhrule

\outputparamWithUnderscore{particleFlux\_vE\_psiHat}
{particleFlux_vE_psiHat}
{\parlink{Nspecies}~$\times$~\parlink{NIterations} matrix of reals}
{\parlink{RHSMode} == 1, \parlink{solveSystem} == \true and \parlink{includePhi1} == \true}
{$\displaystyle \text{\ttfamily particleFlux\_vE\_psiHat} = \frac{\bar{R}}{\bar{n} \bar{v}} \left\langle \int d^3v f_s \, \bm{v}_{E} \cdot \nabla \hat{\psi}\right\rangle$ for all plasma species. The final result is in the last index corresponding to the last nonlinear iteration.}

\myhrule

\outputparamWithUnderscore{particleFlux\_vm0\_psiHat}
{particleFlux_vm0_psiHat}
{\parlink{Nspecies}~$\times$~\parlink{NIterations} matrix of reals}
{\parlink{RHSMode} == 1 and \parlink{solveSystem} == \true}
{$\displaystyle \text{\ttfamily particleFlux\_vm0\_psiHat} = \frac{\bar{R}}{\bar{n} \bar{v}} \left\langle \int d^3v f_{s0} \, \bm{v}_{ms} \cdot \nabla \hat{\psi}\right\rangle$ for all plasma species. The final result is in the last index corresponding to the last nonlinear iteration. This variable should cancel \underscoreparlink{particleFlux\_vE0\_psiHat}{particleFlux_vE0_psiHat} in the last nonlinear iteration since $f_{s0}$ cannot drive radial particle transport (in a calculation without $\Phi_1$ this quantity should be 0 to high precision unless there is a radial current in the magnetic equilibrium),\newline i.e. $\displaystyle \text{\ttfamily particleFlux\_vm0\_psiHat} = - \text{\ttfamily particleFlux\_vE0\_psiHat}$.}

\myhrule

\outputparamWithUnderscore{particleFlux\_vE0\_psiHat}
{particleFlux_vE0_psiHat}
{\parlink{Nspecies}~$\times$~\parlink{NIterations} matrix of reals}
{\parlink{RHSMode} == 1, \parlink{solveSystem} == \true and \parlink{includePhi1} == \true}
{$\displaystyle \text{\ttfamily particleFlux\_vE0\_psiHat} = \frac{\bar{R}}{\bar{n} \bar{v}} \left\langle \int d^3v f_{s0} \, \bm{v}_{E} \cdot \nabla \hat{\psi}\right\rangle$ for all plasma species. The final result is in the last index corresponding to the last nonlinear iteration. This variable should cancel \underscoreparlink{particleFlux\_vm0\_psiHat}{particleFlux_vm0_psiHat} in the last nonlinear iteration since $f_{s0}$ cannot drive radial particle transport, i.e. $\displaystyle \text{\ttfamily particleFlux\_vm0\_psiHat} = - \text{\ttfamily particleFlux\_vE0\_psiHat}$.}

\myhrule

\outputparamWithUnderscore{particleFlux\_vd1\_psiHat}
{particleFlux_vd1_psiHat}
{\parlink{Nspecies}~$\times$~\parlink{NIterations} matrix of reals}
{\parlink{RHSMode} == 1, \parlink{solveSystem} == \true and \parlink{includePhi1} == \true}
{$\displaystyle \text{\ttfamily particleFlux\_vd1\_psiHat} = \text{\ttfamily particleFlux\_vm\_psiHat}~+~\text{\ttfamily particleFlux\_vE0\_psiHat}$. The rationale for defining this quantity is that (at least for $E_r = 0$) it gives the particle flux that would be expected if \parlink{includePhi1} == \false.}

\myhrule

\outputparamWithUnderscore{classicalParticleFlux\_psiHat}
{classicalParticleFlux_psiHat}
{%$\text{\ttfamily Nspecies} \times \text{\ttfamily NIterations}$ 
\parlink{Nspecies}~$\times$~\parlink{NIterations} matrix of reals}
{\parlink{RHSMode} == 1 and \parlink{solveSystem} == \true}
{$\displaystyle \text{\ttfamily classicalParticleFlux\_psiHat} = \frac{\bar{R}}{\bar{n} \bar{v}} \frac{1}{Z_s e} \left\langle \frac{\vect{B} \times \nabla \hat{\psi}}{B^2} \cdot \vect{R}_s \right\rangle$ for all plasma species. $\displaystyle \vect{R}_s = m_s \sum_a \int d^3v \, \vect{v} C_s \left[f_s, f_a\right]$ is the friction force acting on the species. The final result is in the last index corresponding to the last nonlinear iteration.}

\myhrule

\outputparamWithUnderscore{particleFlux\_v\textcolor{magenta}{XN}\_\textcolor{magenta}{Y}}
{particleFlux_vXN_Y}
{\parlink{Nspecies}~$\times$~\parlink{NIterations} matrix of reals}
{See above for various $\textcolor{magenta}{XN}$}
{$\displaystyle \text{\ttfamily particleFlux\_v\textcolor{magenta}{XN}\_\textcolor{magenta}{Y}} = \frac{\bar{R}}{\bar{n} \bar{v}} \left\langle \int d^3v f_{s\textcolor{magenta}{N}} \, \bm{v}_{\textcolor{magenta}{X}s} \cdot \nabla \textcolor{magenta}{Y}\right\rangle$ for all plasma species. The final result is in the last index corresponding to the last nonlinear iteration.\newline
For different options of $\textcolor{magenta}{XN}$ see output {\ttfamily particleFlux} parameters defined above.\newline 
For different options of $\textcolor{magenta}{Y}$ see flux-surface labels (section~\ref{sec:radialCoordinates}).}

\myhrule

\outputparamWithUnderscore{classicalParticleFlux\_\textcolor{magenta}{Y}}
{classicalParticleFlux_Y}
{\parlink{Nspecies}~$\times$~\parlink{NIterations} matrix of reals}
{\parlink{RHSMode} == 1 and \parlink{solveSystem} == \true}
{$\displaystyle \text{\ttfamily classicalParticleFlux\_\textcolor{magenta}{Y}} = \frac{\bar{R}}{\bar{n} \bar{v}} \frac{1}{Z_s e} \left\langle \frac{\vect{B} \times \nabla \textcolor{magenta}{Y}}{B^2} \cdot \vect{R}_s \right\rangle$ for all plasma species. $\displaystyle \vect{R}_s = m_s \sum_a \int d^3v \, \vect{v} C_s \left[f_s, f_a\right]$ is the friction force acting on the species. The final result is in the last index corresponding to the last nonlinear iteration.\newline
For different options of $\textcolor{magenta}{Y}$ see flux-surface labels (section~\ref{sec:radialCoordinates}).}

\myhrule

\outputparamWithUnderscore{classicalParticleFluxNoPhi1\_\textcolor{magenta}{Y}}
{classicalParticleFluxNoPhi1_Y}
{\parlink{Nspecies} array of reals}
{\parlink{RHSMode} == 1}
{$\displaystyle \text{\ttfamily classicalParticleFluxNoPhi1\_\textcolor{magenta}{Y}} = \frac{\bar{R}}{\bar{n} \bar{v}} \frac{1}{Z_s e} \left\langle \frac{\vect{B} \times \nabla \textcolor{magenta}{Y}}{B^2} \cdot \vect{R}_{s,\,\mathrm{w/o}\,\Phi_1} \right\rangle$ for all plasma species. $\displaystyle \vect{R}_{s,\,\mathrm{w/o}\,\Phi_1} = m_s \sum_a \int d^3v \, \vect{v} C_s \left[f_{s,\,\mathrm{w/o}\,\Phi_1}, f_{a,\,\mathrm{w/o}\,\Phi_1}\right]$ is the friction force acting on the species, where the impact of $\Phi_1$ is neglected.\newline
For different options of $\textcolor{magenta}{Y}$ see flux-surface labels (section~\ref{sec:radialCoordinates}).}

\myhrule

%\outputparamWithUnderscore{momentumFlux\_vm\_psiHat}
%{momentumFluxvmpsiHat}
%{$\text{\ttfamily Nspecies} \times \text{\ttfamily NIterations}$ matrix of reals}
%{Always}
%{$\displaystyle \text{\ttfamily momentumFlux\_vm\_psiHat} = \frac{\bar{R}}{\bar{n} \bar{v}^2 \bar{B} \bar{m}} \left\langle \int d^3v f_s \, B m_s v_{\|} \bm{v}_{ms} \cdot \nabla \hat{\psi}\right\rangle$ for all plasma species. The final result is in the last index corresponding to the last nonlinear iteration.}
%
%\myhrule

\outputparamWithUnderscore{momentumFlux\_v\textcolor{magenta}{XN}\_\textcolor{magenta}{Y}}
{momentumFlux_vXN_Y}
{\parlink{Nspecies}~$\times$~\parlink{NIterations} matrix of reals}
{See above for various $\textcolor{magenta}{XN}$}
{$\displaystyle \text{\ttfamily momentumFlux\_v\textcolor{magenta}{XN}\_\textcolor{magenta}{Y}} = \frac{\bar{R}}{\bar{n} \bar{v}^2 \bar{B} \bar{m}} \left\langle \int d^3v f_{s\textcolor{magenta}{N}} \, B m_s v_{\|} \bm{v}_{\textcolor{magenta}{X}s} \cdot \nabla \textcolor{magenta}{Y}\right\rangle$ for all plasma species. The final result is in the last index corresponding to the last nonlinear iteration.\newline
For different options of $\textcolor{magenta}{XN}$ see  
\underscoreparlink{particleFlux\_v\textcolor{magenta}{XN}\_\textcolor{magenta}{Y}}{particleFlux_vXN_Y}.\newline
For different options of $\textcolor{magenta}{Y}$ see flux-surface labels (section~\ref{sec:radialCoordinates}).}

\myhrule

\outputparamWithUnderscore{heatFlux\_v\textcolor{magenta}{XN}\_\textcolor{magenta}{Y}}
{heatFlux_vXN_Y}
{\parlink{Nspecies}~$\times$~\parlink{NIterations} matrix of reals}
{See above for various $\textcolor{magenta}{XN}$}
{$\displaystyle \text{\ttfamily heatFlux\_v\textcolor{magenta}{XN}\_\textcolor{magenta}{Y}} = \frac{\bar{R}}{\bar{n} \bar{v}^3 \bar{m}} \left\langle \int d^3v f_{s\textcolor{magenta}{N}} \, \frac{m_s v^2}{2} \bm{v}_{\textcolor{magenta}{X}s} \cdot \nabla \textcolor{magenta}{Y}\right\rangle$ for all plasma species. The final result is in the last index corresponding to the last nonlinear iteration.\newline
For different options of $\textcolor{magenta}{XN}$ see  
\underscoreparlink{particleFlux\_v\textcolor{magenta}{XN}\_\textcolor{magenta}{Y}}{particleFlux_vXN_Y}.\newline
For different options of $\textcolor{magenta}{Y}$ see flux-surface labels (section~\ref{sec:radialCoordinates}).}

\myhrule

\outputparamWithUnderscore{classicalHeatFlux\_\textcolor{magenta}{Y}}
{classicalHeatFlux_Y}
{\parlink{Nspecies}~$\times$~\parlink{NIterations} matrix of reals}
{\parlink{RHSMode} == 1 and \parlink{solveSystem} == \true}
{\todo{Can Stefan update the text so it is correct?}\newline
$\displaystyle \text{\ttfamily classicalHeatFlux\_\textcolor{magenta}{Y}} = \frac{\bar{R}}{\bar{n} \bar{v}} \frac{1}{Z_s e} \left\langle \frac{\vect{B} \times \nabla \textcolor{magenta}{Y}}{B^2} \cdot \vect{R}_s \right\rangle$ for all plasma species. $\displaystyle \vect{R}_s = m_s \sum_a \int d^3v \, \vect{v} C_s \left[f_s, f_a\right]$ is the friction force acting on the species. The final result is in the last index corresponding to the last nonlinear iteration.\newline
For different options of $\textcolor{magenta}{Y}$ see flux-surface labels (section~\ref{sec:radialCoordinates}).}

\myhrule

\outputparamWithUnderscore{classicalHeatFluxNoPhi1\_\textcolor{magenta}{Y}}
{classicalHeatFluxNoPhi1_Y}
{\parlink{Nspecies} array of reals}
{\parlink{RHSMode} == 1}
{\todo{Can Stefan update the text so it is correct?}\newline
$\displaystyle \text{\ttfamily classicalHeatFluxNoPhi1\_\textcolor{magenta}{Y}} = \frac{\bar{R}}{\bar{n} \bar{v}} \frac{1}{Z_s e} \left\langle \frac{\vect{B} \times \nabla \textcolor{magenta}{Y}}{B^2} \cdot \vect{R}_{s,\,\mathrm{w/o}\,\Phi_1} \right\rangle$ for all plasma species. $\displaystyle \vect{R}_{s,\,\mathrm{w/o}\,\Phi_1} = m_s \sum_a \int d^3v \, \vect{v} C_s \left[f_{s,\,\mathrm{w/o}\,\Phi_1}, f_{a,\,\mathrm{w/o}\,\Phi_1}\right]$ is the friction force acting on the species, where the impact of $\Phi_1$ is neglected.\newline
For different options of $\textcolor{magenta}{Y}$ see flux-surface labels (section~\ref{sec:radialCoordinates}).}

\myhrule

\outputparam{flow}
{\parlink{Nzeta}~$\times$~\parlink{Ntheta}~$\times$~\parlink{Nspecies}~$\times$~\parlink{NIterations} matrix of reals}
{\parlink{RHSMode} == 1 and \parlink{solveSystem} == \true}
{$\displaystyle \text{\ttfamily flow} =  \frac{1}{\bar{n} \bar{v}} \int d^3v \, v_{\|} f_s$.}

\myhrule

\outputparam{velocityUsingFSADensity}
{\parlink{Nzeta}~$\times$~\parlink{Ntheta}~$\times$~\parlink{Nspecies}~$\times$~\parlink{NIterations} matrix of reals}
{\parlink{RHSMode} == 1 and \parlink{solveSystem} == \true}
{{\ttfamily velocityUsingFSADensity} =  \parlink{flow}~/~\parlink{nHats}.}

\myhrule

\outputparam{velocityUsingTotalDensity}
{\parlink{Nzeta}~$\times$~\parlink{Ntheta}~$\times$~\parlink{Nspecies}~$\times$~\parlink{NIterations} matrix of reals}
{\parlink{RHSMode} == 1 and \parlink{solveSystem} == \true}
{{\ttfamily velocityUsingTotalDensity} =  \parlink{flow}~/~\parlink{totalDensity}.}

\myhrule

\outputparam{MachUsingFSAThermalSpeed}
{\parlink{Nzeta}~$\times$~\parlink{Ntheta}~$\times$~\parlink{Nspecies}~$\times$~\parlink{NIterations} matrix of reals}
{\parlink{RHSMode} == 1 and \parlink{solveSystem} == \true}
{{\ttfamily MachUsingFSAThermalSpeed} =  (\parlink{mHats}~/~\parlink{THats})${}^{1/2}$~$\cdot$~\parlink{flow}~/~\parlink{nHats}.}

\myhrule

\outputparam{FSABFlow}
{\parlink{Nspecies}~$\times$~\parlink{NIterations} matrix of reals}
{\parlink{RHSMode} == 1 and \parlink{solveSystem} == \true}
{$\displaystyle \text{\ttfamily FSABFlow} =  \frac{1}{\bar{v} \bar{B} \bar{n}} \left\langle B \int d^3v \, v_{\|} f_s \right\rangle$.}

\myhrule

\outputparam{FSABVelocityUsingFSADensity}
{\parlink{Nspecies}~$\times$~\parlink{NIterations} matrix of reals}
{\parlink{RHSMode} == 1 and \parlink{solveSystem} == \true}
{{\ttfamily FSABVelocityUsingFSADensity} =  \parlink{FSABFlow}~/~\parlink{nHats}.}

\myhrule

\outputparam{FSABVelocityUsingFSADensityOverB0}
{\parlink{Nspecies}~$\times$~\parlink{NIterations} matrix of reals}
{\parlink{RHSMode} == 1 and \parlink{solveSystem} == \true}
{{\ttfamily FSABVelocityUsingFSADensityOverB0} =  ($\bar{B}$~/~$B_0$)~$\cdot$~\parlink{FSABFlow}~/~\parlink{nHats}.}

\myhrule

\outputparam{FSABVelocityUsingFSADensityOverRootFSAB2}
{\parlink{Nspecies}~$\times$~\parlink{NIterations} matrix of reals}
{\parlink{RHSMode} == 1 and \parlink{solveSystem} == \true}
{{\ttfamily FSABVelocityUsingFSADensityOverRootFSAB2} =  (1~/~\parlink{nHats})~$\cdot$~\parlink{FSABFlow}~/~(\parlink{FSABHat2})${}^{1/2}$.}

\myhrule

\outputparam{jHat}
{\parlink{Nzeta}~$\times$~\parlink{Ntheta}~$\times$~\parlink{NIterations} matrix of reals}
{\parlink{RHSMode} == 1 and \parlink{solveSystem} == \true}
{$\displaystyle \text{\ttfamily jHat} = \frac{\bm{j} \cdot \bm{b}}{e \bar{n} \bar{v}} = \frac{1}{e \bar{n} \bar{v}} \sum_s Z_s e \int d^3v \, v_{\|} f_s = \sum_s Z_s \cdot \text{\parlink{flow}}_s$.}

\myhrule

\outputparam{FSABjHat}
{\parlink{NIterations} array of reals}
{\parlink{RHSMode} == 1 and \parlink{solveSystem} == \true}
{$\displaystyle \text{\ttfamily FSABjHat} =  \sum_s Z_s \cdot \text{\parlink{FSABFlow}}_s$.}

\myhrule

\outputparam{FSABjHatOverB0}
{\parlink{NIterations} array of reals}
{\parlink{RHSMode} == 1 and \parlink{solveSystem} == \true}
{$\displaystyle \text{\ttfamily FSABjHatOverB0} =  \frac{\bar{B}}{B_0} \text{\parlink{FSABjHat}}$.}

\myhrule

\outputparam{FSABjHatOverRootFSAB2}
{\parlink{NIterations} array of reals}
{\parlink{RHSMode} == 1 and \parlink{solveSystem} == \true}
{$\displaystyle \text{\ttfamily FSABjHatOverRootFSAB2} =  \frac{\bar{\text{\parlink{FSABjHat}}}}{\left(\text{\parlink{FSABHat2}}\right)^{1/2}}$.}

\myhrule

\section{Distributions}
\label{sec:Distributions}

\myhrule

\outputparam{theta}
{\parlink{Ntheta} array of reals}
{Always}
{Grid points in the poloidal angle, which runs from $0$ to $2 \pi$.}

\myhrule

\outputparam{zeta}
{\parlink{Nzeta} array of reals}
{Always}
{Grid points in the toroidal angle, which runs from $0$ to $2 \pi/$ \parlink{NPeriods}.}

\myhrule

\outputparam{x}
{\parlink{Nx} array of reals}
{Always}
{Grid points in normalized speed, $x_s = v / \sqrt{2 T_s / m_s}$, the same for each species $s$.}

\myhrule

\outputparam{densityPerturbation}
{\parlink{Nzeta}~$\times$~\parlink{Ntheta}~$\times$~\parlink{Nspecies}~$\times$~\parlink{NIterations} matrix of reals}
{\parlink{RHSMode} == 1 and \parlink{solveSystem} == \true}
{$\displaystyle \text{\ttfamily densityPerturbation} =  \frac{1}{\bar{n}} \int d^3v f_{s1}$.}

\myhrule

\outputparam{totalDensity}
{\parlink{Nzeta}~$\times$~\parlink{Ntheta}~$\times$~\parlink{Nspecies}~$\times$~\parlink{NIterations} matrix of reals}
{\parlink{RHSMode} == 1 and \parlink{solveSystem} == \true}
{{\ttfamily totalDensity} = \parlink{nHats}$\cdot \exp \left(- \frac{Z e \Phi_1}{T}  \right)$~+~\parlink{densityPerturbation}.}

\myhrule

\outputparam{FSADensityPerturbation}
{\parlink{Nspecies}~$\times$~\parlink{NIterations} matrix of reals}
{\parlink{RHSMode} == 1 and \parlink{solveSystem} == \true}
{$\displaystyle \text{\ttfamily FSADensityPerturbation} =  \frac{1}{\bar{n}} \left\langle \int d^3v f_{s1} \right\rangle$. 
Should be %within machine precision of 
nearly 0, within roundoff error.}

\myhrule

\outputparam{pressurePerturbation}
{\parlink{Nzeta}~$\times$~\parlink{Ntheta}~$\times$~\parlink{Nspecies}~$\times$~\parlink{NIterations} matrix of reals}
{\parlink{RHSMode} == 1 and \parlink{solveSystem} == \true}
{$\displaystyle \text{\ttfamily densityPerturbation} =  \frac{1}{\bar{n} \bar{T}} \frac{m_s}{3} \int d^3v \, v^2 f_{s1}$.}

\myhrule

\outputparam{totalPressure}
{\parlink{Nzeta}~$\times$~\parlink{Ntheta}~$\times$~\parlink{Nspecies}~$\times$~\parlink{NIterations} matrix of reals}
{\parlink{RHSMode} == 1 and \parlink{solveSystem} == \true}
{{\ttfamily totalPressure} = \parlink{nHats}$\cdot \exp \left(- \frac{Z e \Phi_1}{T}  \right)$ $\cdot$ \parlink{THats}~+~\parlink{pressurePerturbation}.}

\myhrule

\outputparam{FSAPressurePerturbation}
{\parlink{Nspecies}~$\times$~\parlink{NIterations} matrix of reals}
{\parlink{RHSMode} == 1 and \parlink{solveSystem} == \true}
{$\displaystyle \text{\ttfamily FSADensityPerturbation} =  \frac{1}{\bar{n} \bar{T}} \frac{m_s}{3} \left\langle \int d^3v \, v^2 f_{s1} \right\rangle$. 
Should be %within machine precision of 
nearly 0, within roundoff error.}

\myhrule

\outputparamWithUnderscore{delta\_f}
{delta_f}
{$M_x \times M_{\xi} \times M_{\zeta} \times M_{\theta} \times$~\parlink{Nspecies}~$\times$~\parlink{NIterations} matrix of reals}
{\parlink{export\_delta\_f} == \true}
{First order distribution function $f_{s1} = f_{s} - f_{s0}$ for each species, normalized by $\bar{n} / \bar{v}^3$. 
The phase-space grid $\left(M_x, M_{\xi}, M_{\zeta}, M_{\theta}\right)$} is specified by the input parameters in the {\ttfamily \hyperref[sec:exportfParameters]{export\_f}} namelist. 

\myhrule

\outputparamWithUnderscore{full\_f}
{full_f}
{$M_x \times M_{\xi} \times M_{\zeta} \times M_{\theta} \times$~\parlink{Nspecies}~$\times$~\parlink{NIterations} matrix of reals}
{\parlink{export\_full\_f} == \true}
{Full distribution function $f_{s}$ for each species, normalized by $\bar{n} / \bar{v}^3$. 
The phase-space grid $\left(M_x, M_{\xi}, M_{\zeta}, M_{\theta}\right)$} is specified by the input parameters in the {\ttfamily \hyperref[sec:exportfParameters]{export\_f}} namelist. 

\myhrule

\outputparam{Phi1Hat}
{\parlink{Nzeta}~$\times$~\parlink{Ntheta}~$\times$~\parlink{NIterations} matrix of reals}
{\parlink{RHSMode} == 1 and \parlink{solveSystem} == \true and \parlink{includePhi1} == \true}
{Normalized electrostatic potential minus its flux-surface-average, $\displaystyle \text{\ttfamily Phi1Hat} =  \Phi_1/\bar{\Phi}$ where $\Phi_1 = \Phi-\left<\Phi\right>$.}

\myhrule

\outputparam{sources}
{$N_{\mathrm{sources}}$~$\times$~\parlink{Nspecies}~$\times$~\parlink{NIterations} matrix of reals}
{\parlink{RHSMode} == 1 and \parlink{solveSystem} == \true and \parlink{constraintScheme} != 0}
{\todo{Need help from Matt to understand what is in the output. E.g. is the output the coefficients $a_0$, $a_2$, $a_4$ in \parlink{constraintScheme}} }

\myhrule

\section{Miscellaneous}
\label{sec:Miscellaneous}

\myhrule

\outputparam{VPrimeHat}
{real}
{Always}
{$\displaystyle \text{\ttfamily VPrimeHat} =  \int_{0}^{2 \pi} d \theta \int_{0}^{2 \pi} d \zeta \frac{\bar{B}}{\bar{R}} \frac{1}{\nabla \psi \cdot \nabla \theta \times \nabla \zeta}$.}

\myhrule

\outputparam{FSABHat2}
{real}
{Always}
{$\displaystyle \text{\ttfamily FSABHat2} =  \frac{1}{\bar{B}^2} \left\langle B^2 \right\rangle$.}

\myhrule

\outputparam{NPeriods}
{positive integer}
{Always}
{Number of identical toroidal periods (e.g. 5 for W7-X, 10 for LHD, 4 for HSX), equivalent to the \vmec~variable nfp.}

\myhrule

\outputparam{Nspecies}
{positive integer}
{Always}
{Number of particle species}

\myhrule

%The default values are usually best for the parameters in this namelist.
%
%\myhrule
%
%\param{outputFileName}
%{string}
%{``sfincsOutput.h5''}
%{Always}
%{Name which will be used for the HDF5 output file.  If this parameter is changed from the default value, \sfincsScan~ will not work.}
%
%\myhrule

