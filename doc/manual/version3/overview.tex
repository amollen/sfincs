\chapter{Overview}


The \sfincs~code is a freely available, open-source tool for solving neoclassical-type kinetic problems in nonaxisymmetric or axisymmetric plasmas
with nested toroidal flux surfaces.
As with other neoclassical codes, the input information used by \sfincs~is the equilibrium magnetic geometry together with the density, radial density gradient, temperature,
and radial temperature gradient of each species.  The code then solves a drift-kinetic equation for each species,
yielding the (gyro-angle averaged) distribution function.  Moments of the distribution function are computed such as the parallel
flow, bootstrap current, radial particle flux, radial heat flux, and variation of the density over a flux surface.  
These moments are all saved in the output file, and if you wish, you can also save the distribution function itself.
Optionally, a quasi-neutrality
equation can be solved at the same time as the drift-kinetic equations, yielding the self-consistent variation of the electrostatic potential on a flux surface.

The kinetic equations solved in \sfincs~have four independent variables: poloidal angle $\theta$, toroidal angle $\zeta$,
normalized speed $x = v / v_{thermal}$, and pitch angle $\xi = v_{||}/v$.  The third velocity coordinate (gyro-angle) does not appear
since gyro-averaged equations are solved.  The flux surface label (radius) coordinate is only a parameter, rather than a full independent variable,
since a radially local approximation is made. 
The solution is the first-order distribution function $f_{s1} \left(\theta, \zeta, x, \xi\right)$ (where the full distribution is $f_s = f_{s0} + f_{s1}$) on the flux surface for all species $s$, and if quasi-neutrality is solved also the electrostatic potential variation on the flux surface $\Phi_1 \left(\theta, \zeta\right) = \Phi-\left<\Phi\right>$. 

This document discusses the practical use and operation of the code.  For more details about the specific equations implemented,
see the version 3 technical documentation available in the \path{sfincs/docs} directory.
Ref \cite{sfincsPaper} gives many details and some early physics results.

Often, the limiting factor for \sfincs~is the ability of the libraries \mumps~or \superludist~to factorize the preconditioner matrix, discussed in section \ref{sec:gmres}.
You may therefore find it useful to see the control parameters and error codes in the \mumps~user manual:
\url{http://mumps.enseeiht.fr/doc/userguide_5.0.0.pdf}.

This manual describes ``version 3'' of \sfincs.  
To preserve previous versions of the code that have been used for publications, two older versions of the
code called singleSpecies and multiSpecies are also present in the repository.  
For all versions, both MATLAB and fortran editions exist which are independent of each other.
The MATLAB editions exist primarily for debugging the fortran editions.
The same algorithms are implemented in the two different languages, and for identical input parameters, 
the matrices and output quantities from the different editions should agree to several significant digits (roughly within the solver tolerance.)
Any significant differences between the matrices and output of the MATLAB and fortran editions can be used to identify a bug.
The MATLAB versions are serial whereas the fortran versions
are parallelized, so the fortran versions are significantly faster and can access much more memory.
For realistic experimental geometry and collisionality, the resolution (and hence memory) requirements will mean
you will need to use the fortran edition.

\section{Features}

\begin{itemize}

\item
Both self-species and inter-species collisions are treated using the most accurate linear operator available, the full linearized Fokker-Planck collision operator,
with no approximation of the field-particle term or expansion in mass ratio.  This collision operator conserves mass, momentum, and energy.

\item
Realistic experimental geometry can be simulated using an interface to \vmec.  Analytic model equilibria can also be used.

\item
Full coupling in the speed (or equivalently, kinetic energy) coordinate is retained, i.e. no monoenergetic approximation is made.  However,
if desired, \sfincs~can also be run in monoenergetic mode to compare with older codes.

\item
The code is formulated to permit solution of a wide variety of kinetic equations, whether or not phase space volume and/or energy are conserved, so individual terms can be turned on or off to examine their effect. 

\item
A variety of models for terms involving the radial electric field are available to allow comparison between models. 

\item
You can choose to include or not include the poloidal and toroidal magnetic drifts in the kinetic equation.

\item
The code takes advantage of modern algorithms (GMRES) and parallelized libraries (\PETSc, \superludist, and \mumps).

\item
Efficient representation of velocity space is achieved using a pseudospectral method based upon non-classical orthogonal polynomials. \cite{speedGrids}

\item
The electrostatic potential can either be taken to be constant or non-constant on a flux surface.

\item
Optional nonlinear terms in the kinetic equation (involving both the non-Maxwellian distribution function and poloidal/toroidal electric field) can be included using Newton's method. 

\item 
The radial classical particle and heat fluxes are calculated.

\end{itemize}

\section{Limitations}

\begin{itemize}

\item
The \sfincs~code is radially local, in the sense that it approximates 
the radial derivative of the distribution function $\partial f/\partial \psi$ by
the derivative of a Maxwellian flux function $\partial f_M(\psi,x)/\partial \psi$.
This approximation is important for reducing the otherwise 5D space of independent variables $(\psi,\theta,\zeta,x,\xi)$
to a 4D space $(\theta,\zeta,x,\xi)$.
As a result, \sfincs~cannot compute certain finite-orbit-width effects that occur when the radial extent of the particle orbits
between bounces or transits is not small compared to the scale of radial variation in the equilibrium.
Such finite orbit width effects are significant near the magnetic axis, and in strong transport barriers, such as the pedestal of a tokamak H-mode.

\item
Turbulence is neglected.  There are good theoretical reasons to expect that the neoclassical effects computed by \sfincs~should
decouple from turbulence, as detailed in \cite{AbelReview}.  However, this argument relies on an expansion in $\rho_* \ll 1$, and so may break down
in some circumstances when $\rho_*$ is not sufficiently small.

\item
It is assumed that nested toroidal magnetic surfaces exist. Thus, the code cannot accurately model regions of stochastic field,
magnetic islands, or open field lines.

\end{itemize}

\section{Geometry options}
In \sfincs, a variety of options are available for the magnetic field geometry.  The geometry can be read directly
from a {\ttfamily vmec wout}
file, or from the {\ttfamily .bc} format Boozer-coordinate data files used at the Max Planck Institute for Plasma Physics (IPP).
A general analytic model for the magnetic field is also available, given by equation (\ref{eq:Bmodel}),
as are several analytic models for LHD and W7-X in which 3 or 4 Fourier components are retained.
The primary switch for controlling the magnetic geometry in \sfincs~is the \parlink{geometryScheme} parameter
in the {\ttfamily \hyperref[sec:geometryParameters]{geometryParameters}} input namelist.
For more details about geometry options in \sfincs, see section \ref{sec:geometryParameters}.

\section{GMRES/KSP and preconditioning}
\label{sec:gmres}

At its heart, \sfincs~solves one or more large sparse linear systems $Ax=b$.  Here $b$ is a known right-hand side vector,
$A$ is a large (often millions $\times$ millions) known sparse matrix, and $x$ is the desired and unknown solution vector.
The direct way to solve such systems is to $LU$-factorize
the matrix $A$ into lower- and upper-triangular factors.
Once the $L$ and $U$ factors are found, the solution of the linear system
for any right-hand side vector can be rapidly obtained.  However, even if the original matrix is sparse, the $L$ and $U$ factors
are generally not sparse, and so a very large amount of memory can be required for a direct $LU$-factorization.

An alternative way to solve
such large linear systems is with a so-called ``Krylov-space'' iterative method, which can dramatically reduce the memory required
compared to a direct solution.  For the non-symmetric matrices that arise in \sfincs, the preferred Krylov-space
algorithm is called GMRES (Generalized Minimal RESidual.)  In \sfincs, the \PETSc~library is used to solve the large systems
of equations. \PETSc~calls its family of linear solvers KSP, so in the output of \sfincs~you will see a ``KSP residual'' reported
as GMRES iterates towards the solution.

An important element of Krylov methods is preconditioning.  The art of preconditioning is to find a linear operator which has similar eigenvalues
to the ``true'' matrix you would like to invert (or more precisely, to $LU$-factorize), but which can be inverted faster.
If a good preconditioner can be found, the number of GMRES iterations is greatly reduced.  Many schemes for preconditioning exist,
but the version adopted in \sfincs~is to explicity form and $LU$-factorize a preconditioning matrix which is similar
to the true matrix (but somewhat simpler).  There is a basic trade-off:
the more similar the preconditioner matrix is to the true matrix, the fewer iterations will be required, but the more time will
be required to $LU$-factorize the preconditioning operator.  The usual preconditioner matrix in \sfincs~is obtained by dropping all coupling
between grid points in the speed coordinate and dropping coupling between species.  The preconditioner matrix need not be a physically
accurate or meaningful operator; as long as GMRES converges, the solution obtained will be independent of the preconditioner to whatever tolerance is specified.

\section{\sfincs~vs. \sfincsScan}
The core fortran part of \sfincs~solves the kinetic equation for each species
at a single flux surface, a single value of $E_r$, and a single set of other parameters.
However, often the goal is to determine the ambipolar $E_r$ at one or more surfaces, or to scan some other parameter.
For this task, the \sfincsScan~family of \python~scripts is available.
Using these scripts, it is also possible to scan other variables in the input file,
and in particular, to scan the resolution parameters to ensure
the physical output quantities are numerically converged.
For a full list of the types of scans available, see section \ref{sec:sfincsScanParams}

\section{Input and Output}

The input parameters for a \sfincs~computation are specified in a file named {\ttfamily input.namelist}.
This file contains both information for the fortran part of \sfincs~(in standard fortran namelist format),
as well as special lines beginning with {\ttfamily !ss} which are read by \sfincsScan.
The variables which can be specified in {\ttfamily input.namelist} are detailed in chapter \ref{ch:input}.
For scans over minor radius, an additional file named {\ttfamily profiles} is used to specify the profiles of density and temperature
for each species, as well as the range of radial electric field to consider.

The output from a single \sfincs~computation is saved in \HDF~format in the file %{\ttfamily sfincsOutput.h5}.
specified by \parlink{outputFileName}.
To browse this file you can enter {\ttfamily h5dump outputFileName|less} %{\ttfamily h5dump sfincsOutput.h5|less} 
from the command line.
Every array saved in this file is annotated with strings that describe the array dimensions,
for example \Ntheta$\times$ \Nzeta. One of the array dimensions may be {\ttfamily iteration},
which can either indicate the iteration of the Newton solver for a nonlinear calculation,
or which right-hand side vector was used when computing a transport matrix.
Many of the variables in the output file are also annotated with
text that describes their meaning and normalization. 
Some of the most frequently used output quantities are detailed in chapter~\ref{ch:output}.

\section{Questions, Bugs, and Feedback}

We enthusiastically welcome any contributions to the code or documentation.
For write permission to the repository, or to report any bugs, provide feedback, or ask questions, contact Matt Landreman at
\href{mailto:matt.landreman@gmail.com}{\nolinkurl{matt.landreman@gmail.com} }


