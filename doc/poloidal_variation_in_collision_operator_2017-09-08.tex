
\documentclass[12pt]{article}
\usepackage{url} 
%\usepackage[dvips]{graphicx}
\usepackage[pdftex]{graphicx}
\usepackage[latin1]{inputenc}
\usepackage{amsmath}
\usepackage{amssymb}
\usepackage{fancyhdr}
\usepackage{bm}
\usepackage{float}
\usepackage{color}
\usepackage[dvipsnames]{xcolor}
\usepackage{wrapfig}
\usepackage{multicol}
\usepackage[colorlinks=true,
		linkcolor=red,
		citecolor=blue,
		urlcolor=blue]{hyperref}

\usepackage{titlesec}
\usepackage[titletoc]{appendix}
%\usepackage{appendix}
\setcounter{secnumdepth}{4}

\titleformat{\paragraph}
{\normalfont\normalsize\bfseries}{\theparagraph}{1em}{}
\titlespacing*{\paragraph}
{0pt}{3.25ex plus 1ex minus .2ex}{1.5ex plus .2ex}

\setlength {\parindent} { 10mm} 
\setlength{\textheight}{230mm} 
\setlength{\textwidth}{160mm} 
\setlength{\oddsidemargin}{0mm}
\setlength{\topmargin}{-10mm} 
% newcommands
\newcommand{\p}{\partial}
\newcommand{\g}[1]{\mbox{\boldmath $#1$}}
\newcommand{\vi}{\g V_{\! \! i}}
\newcommand{\ps}{Pfirsch-Schl\"{u}ter} 
\newcommand{\lp}{\left(}
\newcommand{\rp}{\right)}
\newcommand{\ca}[1]{\mbox{\cal $#1$}}
\newcommand{\be}{\begin{displaymath}}
\newcommand{\ee}{\end{displaymath}}
\newcommand{\bn}{\begin{equation}}
\newcommand{\en}{\end{equation}}
\newcommand{\mygtrsim}{\mathrel{\mbox{\raisebox{-1mm}{$\stackrel{>}{\sim}$}}}}
\newcommand{\mylsim}{\mathrel{\mbox{\raisebox{-1mm}{$\stackrel{<}{\sim}$}}}}
\newcommand{\vek}{\bf}
\newcommand{\ten}{\sf}
\newcommand{\bfm}[1]{\mbox{\boldmath$#1$}}
\newcommand{\lang}{\left\langle}
\newcommand{\rang}{\right\rangle}
\newcommand{\vo}[1]{\left|\begin {array}{l} \mbox{} \\ \mbox{} \\$#1$ \end
{array}\right .}  
\newcommand{\von}[2]{\left |\begin {array}{l}
\mbox{}\\$#1$\\$#2$ \end {array}\right .}
\newcommand{\simgt}{\:{\raisebox{-1.5mm}{$\stackrel
{\textstyle{>}}{\sim}$}}\:}
\newcommand{\simlt}{\:{\raisebox{-1.5mm}{$\stackrel
{\textstyle{<}}{\sim}$}}\:}
%\renewcommand {\baselinestretch} {1.67}
%\pagestyle{empty}
\newcommand{\todo}[1]{\textbf{\textcolor{red}{TODO: #1}}}
\newcommand{\remark}[1]{\textbf{\textcolor{red}{REMARK: #1}}}

\title{Implementation of poloidal density variation in collision operator}

\pagestyle{fancy}
\fancyhead{}
\chead{Stefan Buller %850227-2019
\\ Implementation of poloidal density variation in collision operator}
\cfoot{\thepage}
\renewcommand{\headrulewidth}{1pt}
\renewcommand{\footrulewidth}{1pt}
\setlength{\headheight}{28pt}
\setlength{\footskip}{25pt}

\newcommand{\red}[1]{\textcolor{red}{#1}}
\newcommand{\mE}{\mathcal{E}}
\newcommand{\energy}{\mathcal{E}}
\newcommand{\mK}{\mathcal{K}}
\newcommand{\mN}{\mathcal{N}}
\newcommand{\mD}{\mathcal{D}}
\newcommand{\ord}{\mathcal{O}}
\newcommand{\Tpe}{T_\perp}
\newcommand{\Tpa}{T_\|}
\newcommand{\vpe}{v_\perp}
\newcommand{\vpa}{v_\|}
\newcommand{\kpe}{k_\perp}
\newcommand{\kpa}{k_\|}
\newcommand{\Bv}{\mathbf{B}}
\newcommand{\Ev}{\mathbf{E}}
\newcommand{\bv}{\mathbf{b}}
\newcommand{\vv}{\mathbf{v}}
\newcommand{\cd}{\cdot}
\newcommand{\na}{\nabla}
\newcommand{\btheta}{\bar{\theta}}
\newcommand{\phit}{\tilde{\phi}}
\newcommand{\oert}{\tilde{\omega}_{Er}}

\begin{document}
\titlepage

\maketitle

These notes are concerned with the implementation of the \texttt{poloidal\-Variation\-In\-Collision\-Operator} option in SFINCS version 3.

\section*{$\Phi_1$ implementation in SFINCS}
In a previous set of notes (\emph{$\Phi_1$ implementation}), poloidal density variation to the lowest order was implemented by linearizing around a poloidally varying Maxwellian $f_0$, rather than a flux-function Maxwellian $f_M$.

Poloidal variations can occur to lowest order if the potential varies on a flux-surface, in which case the lowest order distribution function becomes
\begin{equation}
f_0(\psi,\theta,\zeta) = f_M(\psi) e^{-Ze\Phi_1(\theta,\zeta)/T}.
\end{equation}
[We will in the future suppress the $\psi$ dependence, as SFINCS only solves for a single flux-surface.]

In the \emph{$\Phi_1$ implementation} notes, the equations were modified to linearize around this $f_0$ rather than $f_M$. However, the linearization in the collision operator was not treated, which is the subject of these notes.


\section*{$\Phi_1$ implementation in the linearized collision operator}
The linearized collision operator can be written as
\begin{equation}
C^{L:f_0}_{ab} = C_{ab}\{f_{a0},f_{b0}\} +  C_{ab}\{f_{a1},f_{b0}\} +  C_{ab}\{f_{a0},f_{b1}\},
\end{equation}
where we use the superscript $L:f_0$ to indicate that the operator has been linearized around $f_0$ rather than $f_M$. 

As $f_0$ and $f_M$ have the same velocity space-structure, the terms in the above linearization can easily be expressed in terms of the terms in the linearization around $f_M$:
\begin{equation}
C^{L:f_0}_{ab} =  C_{ab}\{f_{aM},f_{bM}\} e^{-(Z_a/T_a + Z_b/T_b) e\Phi_1(\theta,\zeta)} +  C_{ab}\{f_{a1},f_{b0}\} e^{-Z_b e\Phi_1(\theta,\zeta)/T_b} +  C_{ab}\{f_{aM},f_{b1}\} e^{-Z_a e\Phi_1(\theta,\zeta)/T_a}.
\end{equation}
Effectively, from the explicit expressions for $C^{L:f_M}$, one can obtain $C^{L:f_0}$ by substituting the density of each species by:
\begin{equation}
n_a \to n_a e^{-\frac{Z_a e\Phi_1(\theta,\zeta)}{T_a}}. \label{eq:denssub}
\end{equation}

\subsection*{Changes to the Jacobian}
As the old collision operator only depend on $f_{1}$ (neglecting the temperature equilibration term $C_{ab}\{f_{aM},f_{bM}\}$ for the moment), its contribution to the Jacobian was
\begin{equation}
J^{L:f_M} = \frac{\delta C^{L:f_M}}{\delta f_{1}}.
\end{equation}
Furthermore, as the linearized collision operator is linear in $f_{1}$, we have
\begin{equation}
\frac{\delta C^{L:f_M}}{\delta f_{1}} =  C^{L:f_M}[.],
\end{equation}
where the notation $C^{L:f_M}[.]$ implies that the operator does not act on anything, i.e. the matrix used to represent $C^{L:f_M}[.]$ enters as is into the Jacobian.

In the system linearized around $f_0$, we have essentially the same $f_1$ dependence, if we substitute the density as in \eqref{eq:denssub}
\begin{equation}
\frac{\delta C^{L:f_0}}{\delta f_{1}} =  C^{L:f_M}[.] \left\{n_a \to n_a e^{-\frac{Z_a e\Phi_1(\theta,\zeta)}{T_a}}\right\},
\end{equation}
however, as $\Phi_1$ is an unknown, we get new non-zero elements in the Jacobian due to $\frac{\delta C^{L:f_0}}{\delta \Phi_{1}}$.
As $C^{L:f_0}$ only depends on $\Phi_1$ through $n_a e^{-\frac{Z_a e\Phi_1(\theta,\zeta)}{T_a}}$, we can also get the new expressions through as a density substitution
\begin{equation}
n_a e^{-\frac{Z_a e\Phi_1}{T_a}} \to  -\frac{Z_a e}{T_a} n_a e^{-\frac{Z_a e\Phi_1}{T_a}}.
\end{equation}
Thus
\begin{equation}
\frac{\delta C^{L:f_0}}{\delta \Phi_{1}} =  C^{L:f_M}[f_1] \left\{n_a \to -\frac{Z_a e}{T_a} n_a e^{-\frac{Z_a e\Phi_1}{T_a}}\right\}.
\end{equation}
Note that the collision operator here acts on $f_1$.

In addition, if we include temperature equilibration, we get an additional contribtution to the Jacobian
\begin{equation}
\frac{\delta C^{L:f_0}}{\delta \Phi_{1}} =  -e\frac{Z_a T_b + Z_b T_a}{T_a T_b} e^{\left(-\frac{Z_a T_b + Z_b T_a}{T_a T_b} e\Phi_1\right)} C_{ab}\{f_{aM},f_{bM}\}.
\end{equation}
This term, like all the other in this document, is merely an extra factor to an expression already calculated in the code.

\section*{Changes to the code}
The collision operator is linearized around $f_0$ if \texttt{poloidalVariationInCollisionOperator} is set to true in the input file. This switch adds the extra factor $e^{-\frac{Z_a e\Phi_1}{T_a}}$ to the densities in the collision operator as calculated in the code, and reuses the calculations to evaluate $\frac{\delta C^{L:f_0}}{\delta \Phi_{1}}$. If \texttt{includeTemperatureEquilibrationTerm} is set to true, the contribution from this term is also included in the Jacobian. 



\end{document}
